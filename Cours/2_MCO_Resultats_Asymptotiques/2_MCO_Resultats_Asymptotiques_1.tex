\documentclass[11pt, reqno]{amsart}
\numberwithin{equation}{section}
\usepackage[svgnames]{xcolor}
\usepackage{setspace} 
\usepackage{etex}
\usepackage{tikz}
\usepackage{mathrsfs}
\usepackage{todonotes}                         %I use this to include comments on the PDF document. 
                                                                     %Requires todonotes.sty
\usepackage[labelfont=bf, margin=1cm]{caption}

\usepackage{pst-func}
\usepackage[normalem]{ulem}
\usepackage{amsmath}
\usepackage{amsthm}
\usepackage{thmtools}
\usepackage{amsfonts}
\usepackage{amssymb}
\usepackage{graphicx}
\usepackage{hyperref}
\usepackage{tikz}
\usepackage{mdframed}
\usepackage{showexpl}
%\usepackage{geometry}
%\geometry{verbose,tmargin=1.5cm,bmargin=1.5cm,lmargin=1.5cm,rmargin=1.5cm}
%\usepackage[left=1in,right=1in,top=0.9in,bottom=0.9in]{geometry}
%\usepackage[vmargin={1.5in, 1.5in},hmargin={1.5in, 1.5in}]{geometry} 
\usepackage[vmargin={1.5in, 1.5in},hmargin={1.5in, 1.5in}]{geometry} 
%\usepackage[vmargin={1.5in, 1.5in},hmargin={1.2in, 1.2in}]{geometry} 
%\usepackage{geometry}
%\geometry{verbose,tmargin=3cm,bmargin=3cm,lmargin=3cm,rmargin=3cm}
\usepackage{setspace}
%\usepackage[latin9]{luainputenc}
\setlength{\parskip}{\medskipamount}
%\setlength{\parindent}{0pt}
%\documentclass[12pt, reqno]{amsart}
%\usepackage[svgnames]{xcolor}
%\usepackage{setspace} 
%\usepackage{etex}
%\usepackage{tikz}
%\usepackage{todonotes}                         %I use this to include comments on the PDF document. 
                                                                     %Requires todonotes.sty
%\usepackage[labelfont=bf, margin=1cm]{caption}

%\usepackage{pst-func}
%\usepackage[normalem]{ulem}
%\usepackage{amsmath}
%\usepackage{amsthm}
%\usepackage{amsfonts}
%\usepackage{amssymb}
%\usepackage{graphicx}
%\usepackage{hyperref}
%\usepackage[left=1in,right=1in,top=0.9in,bottom=0.9in]{geometry}
%\usepackage[vmargin={1.5in, 1.5in},hmargin={1in, 1in}]{geometry} 
\usepackage{multirow}
\usepackage{verbatim}
\usepackage[greek,frenchb]{babel}
\usepackage[babel=true,kerning=true]{microtype}
\usepackage[utf8]{inputenc}
\usepackage[T1]{fontenc}
\usepackage{lmodern} \normalfont %to load T1lmr.fd 
\DeclareFontShape{T1}{lmr}{bx}{sc} { <-> ssub * cmr/bx/sc }{}
\usepackage{pdfpages}
%\usepackage{enumitem}
\usepackage{enumerate}    
\usepackage{natbib}
\usepackage{ctable}
\usepackage{lscape}
\usepackage{array}
\usepackage{bbm}
\usepackage{hyperref}
\usepackage{graphicx}
\usepackage{epsf}
%\usepackage{fancyhdr}
\usepackage{setspace}
\usepackage{verbatim}
%\usepackage{courier}
\usepackage{color}
\usepackage{hyperref}
\usepackage{fncylab}
\usepackage{authblk}
\usepackage{textcomp}
%\usepackage{tmmath} %_   (TM-Math/Times)
%\usepackage{cmbright}
%\usepackage{ccfonts} \usepackage[T1]{fontenc}
%\usepackage[garamond]{mathdesign}
\usepackage{xcolor}
\usepackage[normalem]{ulem}
    

%\usepackage{newpxmath}
%\usepackage{newpxtext}
%\usepackage{sectsty}
\usepackage{lipsum}
\definecolor{ugablue}{cmyk}{1.00,.62,.0,.52}
%\definecolor{ugablue}{rgb}{0.1,0.2,0.65}
\definecolor{ugagrey}{cmyk}{.42,.08,.0,.40}
%\hypersetup{
  %colorlinks   = true, %Colours links instead of ugly boxes
  %urlcolor     = ugablue, %Colour for external hyperlinks
  %linkcolor    = DarkBlue, %Colour of internal links
 % citecolor   = DarkBlue %Colour of citations
%}
\hypersetup{
  colorlinks   = true, %Colours links instead of ugly boxes
  urlcolor     = ugablue, %Colour for external hyperlinks
  linkcolor    = MidnightBlue, %Colour of internal links
  citecolor   = MidnightBlue %Colour of citations
}
\newtheorem{theoreme}{Théorème}[section]
\newtheorem*{theoreme*}{Théorème}
%\newtheorem{conjecture}{Conjecture}[section]
%\newtheorem{corollary}{Corollary}[section]
\newtheorem{lemme}{Lemme}[section]
\newtheorem{proposition}{Proposition}[section]
%\newtheorem{propriete}{Propri\'et\'e}
%\newtheorem{assumption}{Assumption}
%\renewcommand{\theassumption}{A\arabic{assumption}}
%\newtheorem{hypothese}{Hypothèse}
%\renewcommand{\thehypothese}{H\arabic{hypothese}}
%\newtheorem{condition}{Condition}
%\renewcommand{\thecondition}{C\arabic{condition}}
%\newtheorem{propriete}{Propriété}
%\renewcommand{\thepropriete}{P\arabic{propriete}}
%\newtheorem{remarque}{Remarque}
%\theoremstyle{definition}
%\newtheorem{definition}{Définition}
%\labelformat{definition}{D\arabic{definition}}
%\newtheorem{exemple}{Exemple}
%\renewcommand{\thedefinition}{Définition D\arabic{definition}}
\newcommand{\argmax}{\operatornamewithlimits{arg\,max\,}}
\newcommand{\argmin}{\operatornamewithlimits{arg\,min\,}}
\providecommand{\plim}{\underset{n \rightarrow \infty}{\operatornamewithlimits{plim}}}
\def\inprobLOW{\rightarrow_p}
\def\inprobHIGH{\,{\buildrel p \over \rightarrow}\,} 
\def\inprob{\,{\inprobHIGH}\,} 
\def\indist{\,{\buildrel d \over \rightarrow}\,} 
\def\F{\mathbb{F}}
\def\R{\mathbb{R}}
\def\N{\mathbb{N}}
\newcommand{\gmatrix}[1]{\begin{pmatrix} {#1}_{11} & \cdots &
    {#1}_{1n} \\ \vdots & \ddots & \vdots \\ {#1}_{m1} & \cdots &
    {#1}_{mn} \end{pmatrix}}
\newcommand{\iprod}[2]{\left\langle {#1} , {#2} \right\rangle}
\newcommand{\norm}[1]{\left\Vert {#1} \right\Vert}
\newcommand{\abs}[1]{\left\vert {#1} \right\vert}
\renewcommand{\det}{\mathrm{det}}
%\newcommand{\rank}{\mathrm{rank}}
\newcommand{\spn}{\mathrm{span}}
\newcommand{\row}{\mathrm{Row}}
\newcommand{\col}{\mathrm{Col}}
\renewcommand{\dim}{\mathrm{dim}}
\newcommand{\prefeq}{\succeq}
\newcommand{\pref}{\succ}
\newcommand{\seq}[1]{\{{#1}_n \}_{n=1}^\infty }
%\providecommand{\limp}{\underset{n \rightarrow \infty}{\overset{p}{\longrightarrow}}}
%\providecommand{\limp}{\underset{n \rightarrow +\infty}{\overset{p.}{\longrightarrow}}}
%\providecommand{\limp}{\overset{p.}{\longrightarrow}}
\renewcommand{\to}{\rightarrow}
\providecommand{\limp}{\overset{p}{\rightarrow}}
%\providecommand{\limp}{\overset{p}{\to}}
\providecommand{\limps}{\overset{p.s.}{\rightarrow}}
%\providecommand{\limps}{\overset{p.s.}{\to}}
\providecommand{\limd}{\overset{d}{\rightarrow}}
\providecommand{\limlp}{\overset{L^p}{\rightarrow}}
%\providecommand{\limd}{\overset{d}{\to}}
%\providecommand{\limps}{\underset{n \rightarrow +\infty}{\overset{p. s.}{\longrightarrow}}}
%\providecommand{\limp}{\overset{p}{\longrightarrow}}
%\providecommand{\limd}{\underset{n \rightarrow +\infty}{\overset{d.}{\longrightarrow}}}
\providecommand{\approxdist}{\overset{\mathrm{a}}{\sim}}
\providecommand{\sima}{\overset{\mathrm{a}}{\sim}}
%\providecommand{\limd}{\overset{d.}{\longrightarrow}}
%\newcommand{\sumobs}{\underset{i=1}{\overset{n}{\sum}}}
\newcommand{\sumobs}{\sum_{i=1}^n}
\newcommand{\sumiton}{\sum_{i=1}^n}
\newcommand{\sumi}{\underset{i=1}{\overset{n}{\sum}}}
\newcommand{\prodi}{\underset{i=1}{\overset{n}{\prod}}}
\newcommand{\sumt}{\underset{t=1}{\overset{T}{\sum}}}
\newcommand{\sumti}{\underset{t=1}{\overset{T_i}{\sum}}}
\newcommand{\sumitok}{\underset{i=1}{\overset{k}{\sum}}}
\newcommand{\sumjton}{\underset{j=1}{\overset{n}{\sum}}}
\newcommand{\sumttoT}{\underset{t=1}{\overset{T}{\sum}}}
\newcommand{\sumkton}{\underset{k=1}{\overset{n}{\sum}}}
\newcommand{\sumjtoinfty}{\underset{j=1}{\overset{\infty}{\sum}}}
\newcommand{\sumjzerotoinfty}{\underset{j=0}{\overset{\infty}{\sum}}}
\newcommand{\sumkzerotoinfty}{\underset{k=0}{\overset{\infty}{\sum}}}
\newcommand{\sumkonetoinfty}{\underset{k=1}{\overset{\infty}{\sum}}}
\newcommand{\sumlzerotoinfty}{\underset{l=0}{\overset{\infty}{\sum}}}
\newcommand{\summinftoinf}{\underset{-\infty}{\overset{\infty}{\sum}}}
\newcommand{\limntoinfty}{\underset{n\rightarrow\infty}{\lim}}
\newcommand{\prodobs}{\underset{i=1}{\overset{n}{\prod}}}
\newcommand{\sumtoinf}{\underset{i=0}{\overset{\infty}{\sum}}}
%\renewcommand{\to}{{\rightarrow}}
\providecommand{\Lsq}{{\mathcal{L}^2}}
\providecommand{\En}{\mathbb{E}_n}
\providecommand{\Gn}{\mathrm{G}_n}
%\providecommand{\Var}{\mathbf{Var}}
%\providecommand{\Var}{\mathsf{Var}}
%\providecommand{\Var}{\mathrm{Var}}
%\providecommand{\Vr}{\mathbb{V}}
\providecommand{\Er}{\mathbb{E}}
%\providecommand{\Exp}{\mathbf{E}}
%\providecommand{\Exp}{\mathsf{E}}
%\providecommand{\Exp}{\mathbb{E}}
%\providecommand{\Exp}{\mathrm{E}}
%\DeclareMathOperator{\Exp}{\mathrm{E}}
%\DeclareMathOperator{\Var}{\mathrm{Var}}
%\DeclareMathOperator{\Cov}{\mathrm{Cov}}
%\DeclareMathOperator{\corr}{\mathrm{corr}}
\DeclareMathOperator{\indfun}{\mathbf{1}}
\DeclareMathOperator{\Exp}{E}
\DeclareMathOperator{\Expn}{\mathbb{E}_n}
\DeclareMathOperator{\Var}{Var}
\DeclareMathOperator{\Vr}{V}
\DeclareMathOperator{\Avar}{AVar}
\DeclareMathOperator{\Cov}{Cov}
\DeclareMathOperator{\Cv}{Cv}
\DeclareMathOperator{\corr}{corr}
\providecommand{\Cr}{\mathcal{C}}
\DeclareMathOperator{\perps}{\perp_s}
%\DeclareMathOperator{\Prob}{Pr}
\DeclareMathOperator{\Prob}{P}
\DeclareMathOperator{\prob}{p}
\DeclareMathOperator{\loss}{L}
%\DeclareMathOperator{\Exp}{\mathbf{E}}
%\DeclareMathOperator{\Var}{\mathbf{Var}}
%\DeclareMathOperator{\Cov}{\mathbf{Cov}}
%\DeclareMathOperator{\corr}{\mathbf{corr}}
%\DeclareMathOperator{\Prob}{\mathbf{Pr}}
%\providecommand{\Cov}{\mathbf{Cov}}
%\providecommand{\Cov}{\mathsf{Cov}}
%\providecommand{\Cov}{\mathrm{Cov}}
%\providecommand{\corr}{\mathbf{corr}}
\providecommand{\Id}{\mathbf{I}}
\providecommand{\Ind}{\mathbf{1}}
\providecommand{\uvec}{\mathbf{1}}
\providecommand{\vecOnes}{\mathbf{1}}
%\providecommand{\vecOnes}{\mathbf{e}}
\providecommand{\Rang}{\mathrm{Rang}}
\providecommand{\Trace}{\mathrm{Trace}}
%\providecommand{\Tr}{\mathrm{Tr}}
%\providecommand{\Cov}{\mathbb{C}\mathrm{ov}}
\providecommand{\Diag}{\mathrm{Diag}}
\providecommand{\Pred}{\mathcal{P}}
\providecommand{\sp}{\mathrm{span}}
\providecommand{\CI}{\mathrm{CI}}
\providecommand{\reg}{\mathrm{r}}
\providecommand{\Likelihood}{\mathrm{L}}
%\renewcommand{\Pr}{{\mathbb{P}}}
\providecommand{\set}[1]{\left\{#1\right\}}
\providecommand{\asyvar}{\mathrm{AsyVar}}
\providecommand{\dapprox}{\overset{a.}{\approx}}
%\providecommand{\plim}{\operatornamewithlimits{plim}}
\newcommand\indep{\protect\mathpalette{\protect\independenT}{\perp}}
%\newcommand{\independent}{\perp \!\!\! \perp}
\DeclareMathOperator{\indpnt}{\perp \!\!\! \perp}
\def\independenT#1#2{\mathrel{\setbox0\hbox{$#1#2$}%
  \copy0\kern-\wd0\mkern4mu\box0}} 
\providecommand{\iiddat}{\{(y_i, x_i)\}_{i=1}^n}
\newcommand{\dsim}{\stackrel {\textrm{appr }} {\sim} }
%\providecommand{\dsim}{{\overset{appr.}{\sim}}}
%%Commands from Econometric Theory(Slides) by J. Stachurski.
\newcommand\T{{\mathpalette\raiseT\intercal}}
\newcommand\raiseT[2]{\raisebox{0.25ex}{$#1#2$}}
\newcommand{\boldx}{ {\mathbf x} }
\newcommand{\boldu}{ {\mathbf u} }
\newcommand{\boldv}{ {\mathbf v} }
\newcommand{\boldw}{ {\mathbf w} }
\newcommand{\boldy}{ {\mathbf y} }
\newcommand{\boldb}{ {\mathbf b} }
\newcommand{\bolda}{ {\mathbf a} }
\newcommand{\boldc}{ {\mathbf c} }
\newcommand{\boldi}{ {\mathbf i} }
\newcommand{\bolde}{ {\mathbf e} }
\newcommand{\boldp}{ {\mathbf p} }
\newcommand{\boldq}{ {\mathbf q} }
\newcommand{\bolds}{ {\mathbf s} }
\newcommand{\boldt}{ {\mathbf t} }
\newcommand{\boldz}{ {\mathbf z} }
\newcommand{\boldr}{ {\mathbf r} }
\newcommand{\boldd}{ {\mathbf d} }

\newcommand{\boldzero}{ {\mathbf 0} }
\newcommand{\boldone}{ {\mathbf 1} }

\newcommand{\boldalpha}{ {\boldsymbol \alpha} }
\newcommand{\boldbeta}{ {\boldsymbol \beta} }
\newcommand{\boldgamma}{ {\boldsymbol \gamma} }
\newcommand{\boldtheta}{ {\boldsymbol \theta} }
\newcommand{\boldxi}{ {\boldsymbol \xi} }
\newcommand{\boldtau}{ {\boldsymbol \tau} }
\newcommand{\boldepsilon}{ {\boldsymbol \epsilon} }
\newcommand{\boldmu}{ {\boldsymbol \mu} }
\newcommand{\boldSigma}{ {\boldsymbol \Sigma} }
\newcommand{\boldOmega}{ {\boldsymbol \Omega} }
\newcommand{\boldPhi}{ {\boldsymbol \Phi} }
\newcommand{\boldLambda}{ {\boldsymbol \Lambda} }
\newcommand{\boldphi}{ {\boldsymbol \phi} }

\newcommand{\Sigmax}{ {\boldsymbol \Sigma_{\boldx}}}
\newcommand{\Sigmau}{ {\boldsymbol \Sigma_{\boldu}}}
\newcommand{\Sigmaxinv}{ {\boldsymbol \Sigma_{\boldx}^{-1}}}
\newcommand{\Sigmav}{ {\boldsymbol \Sigma_{\boldv \boldv}}}

\newcommand{\hboldx}{ \hat {\mathbf x} }
\newcommand{\hboldy}{ \hat {\mathbf y} }
\newcommand{\hboldb}{ \hat {\mathbf b} }
\newcommand{\hboldu}{ \hat {\mathbf u} }
\newcommand{\hboldtheta}{ \hat {\boldsymbol \theta} }
\newcommand{\hboldtau}{ \hat {\boldsymbol \tau} }
\newcommand{\hboldmu}{ \hat {\boldsymbol \mu} }
\newcommand{\hboldbeta}{ \hat {\boldsymbol \beta} }
\newcommand{\hboldgamma}{ \hat {\boldsymbol \gamma} }
\newcommand{\hboldSigma}{ \hat {\boldsymbol \Sigma} }
\newcommand{\betahat}{\hat{\beta}}
\newcommand{\gammahat}{\hat{\gamma}}
\newcommand{\Ahat}{\hat{A}}
\newcommand{\epsilonhat}{\hat{\epsilon}}
\newcommand{\varepsilonhat}{\hat{\varepsilon}}
\newcommand{\sigmahat}{\hat{\sigma}}
\newcommand{\Sigmahat}{\hat{\Sigma}}
\newcommand{\boldA}{\mathbf A}
\newcommand{\boldB}{\mathbf B}
\newcommand{\boldC}{\mathbf C}
\newcommand{\boldD}{\mathbf D}
\newcommand{\boldG}{\mathbf G}
\newcommand{\boldI}{\mathbf I}
\newcommand{\boldL}{\mathbf L}
\newcommand{\boldM}{\mathbf M}
\newcommand{\boldP}{\mathbf P}
\newcommand{\boldQ}{\mathbf Q}
\newcommand{\boldR}{\mathbf R}
\newcommand{\boldX}{\mathbf X}
\newcommand{\boldU}{\mathbf U}
\newcommand{\boldV}{\mathbf V}
\newcommand{\boldW}{\mathbf W}
\newcommand{\boldY}{\mathbf Y}
\newcommand{\boldZ}{\mathbf Z}

\newcommand{\hboldU}{ \hat {\mathbf U} }
\newcommand{\hboldOmega}{ \hat{{\boldsymbol \Omega} }}
\newcommand{\bSigmaX}{ {\boldsymbol \Sigma_{\hboldbeta}} }
\newcommand{\hbSigmaX}{ \mathbf{\hat \Sigma_{\hboldbeta}} }

\newcommand{\RR}{\mathbbm R}
\newcommand{\CC}{\mathbbm C}
%\newcommand{\NN}{\mathbbm N}
\newcommand{\PP}{\mathbbm P}
\newcommand{\EE}{\mathbbm E \nobreak\hspace{.1em}}
\newcommand{\EEP}{\mathbbm E_P \nobreak\hspace{.1em}}
\newcommand{\ZZ}{\mathbbm Z}
\newcommand{\QQ}{\mathbbm Q}


\newcommand{\XX}{\mathcal X}

\newcommand{\aA}{\mathcal A}
\newcommand{\fF}{\mathscr F}
\newcommand{\bB}{\mathscr B}
\newcommand{\iI}{\mathscr I}
\newcommand{\rR}{\mathscr R}
\newcommand{\dD}{\mathcal D}
\newcommand{\lL}{\mathcal L}
\newcommand{\llL}{\mathcal{H}_{\ell}}
\newcommand{\gG}{\mathcal G}
\newcommand{\hH}{\mathcal H}
\newcommand{\nN}{\textrm{\sc n}}
\newcommand{\lN}{\textrm{\sc ln}}
\newcommand{\pP}{\mathscr P}
\newcommand{\qQ}{\mathscr Q}
\newcommand{\xX}{\mathcal X}

\newcommand{\ddD}{\mathscr D}
\newcommand{\lnorm}{\mathit{l}}


%\newcommand{\R}{{\texttt R}}
\newcommand{\risk}{\mathcal R}
\newcommand{\Remp}{R_{{\rm emp}}}

\newcommand*\diff{\mathop{}\!\mathrm{d}}
\newcommand{\ess}{ \textrm{{\sc ess}} }
\newcommand{\tss}{ \textrm{{\sc tss}} }
\newcommand{\rss}{ \textrm{{\sc rss}} }
\newcommand{\rssr}{ \textrm{{\sc rssr}} }
\newcommand{\ussr}{ \textrm{{\sc ussr}} }
\newcommand{\zdata}{\mathbf{z}_{\mathcal D}}
\newcommand{\Pdata}{P_{\mathcal D}}
\newcommand{\Pdatatheta}{P^{\mathcal D}_{\theta}}
\newcommand{\Zdata}{Z_{\mathcal D}}


\newcommand{\e}[1]{\mathbbm{E}[{#1}]}
\newcommand{\p}[1]{\mathbbm{P}({#1})}



\DeclareMathOperator{\cl}{cl}
%\DeclareMathOperator{\argmax}{argmax}
\DeclareMathOperator{\interior}{int}
%\DeclareMathOperator{\Prob}{Prob}
\DeclareMathOperator{\kernel}{ker}
\DeclareMathOperator{\diag}{diag}
\DeclareMathOperator{\lag}{\mathbf{L}}
\DeclareMathOperator{\AR}{\mathrm{AR}}
\DeclareMathOperator{\MA}{\mathrm{MA}}
\DeclareMathOperator{\ARMA}{\mathrm{ARMA}}
\DeclareMathOperator{\sgn}{sgn}
\DeclareMathOperator{\determinant}{det}
\DeclareMathOperator{\vech}{vech}
\DeclareMathOperator{\trace}{trace}
\DeclareMathOperator{\Tr}{Tr}
\DeclareMathOperator{\Span}{span}
\DeclareMathOperator{\rank}{rank}
%\DeclareMathOperator{\cov}{cov}
%\DeclareMathOperator{\corr}{corr}
\DeclareMathOperator{\range}{rng}
%\DeclareMathOperator{\Var}{Var}
\DeclareMathOperator{\mse}{mse}
\DeclareMathOperator{\se}{se}
%\DeclareMathOperator{\row}{row}
%\DeclareMathOperator{\col}{col}
\DeclareMathOperator{\dimension}{dim}
\DeclareMathOperator{\fracpart}{frac}
\DeclareMathOperator{\proj}{proj}
\DeclareMathOperator{\colspace}{colspace}
%\theoremstyle{plain}
%\newtheorem{axiom}{Axiom}[section]
%\newtheorem{theorem}{Theorem}[section]
%\newtheorem{corollary}{Corollary}[section]
%\newtheorem{lemma}{Lemma}[section]
%\newtheorem{proposition}{Proposition}[section]
%
\theoremstyle{definition}
%\newtheorem{definition}{Definition}[section]
%\newtheorem{example}{Example}[section]
%\newtheorem{remark}{Remark}[section]
%\newtheorem{notation}{Notation}[section]
\newtheorem{assumption}{Assumption}
\renewcommand{\theassumption}{A\arabic{assumption}}
%\newtheorem{condition}{Condition}[section]
%\newtheorem{exercise}{Ex.}[section]
%\newtheorem{fact}{Fact}[section]
%%
\newtheoremstyle{interpret}% name
{12pt}% Space above
{5pt}% Space below
{}% Body font
{}% Indent amount
{\bfseries}% Theorem head font
{:}% Punctuation after theorem head
{.3em}% Space after theorem head
{}% Theorem head spec (can be left empty, meaning `normal')

\theoremstyle{interpret}
\newtheorem{interpret}{\textsc{Interprétation}}
\renewcommand{\theinterpret}{}

%\newtheoremstyle{interpretation}% name
%{12pt}% Space above
%{5pt}% Space below
%{}% Body font
%{}% Indent amount
%{\bfseries}% Theorem head font
%{}% Punctuation after theorem head
%{.5em}% Space after theorem head
%{}% Theorem head spec (can be left empty, meaning `normal')


\theoremstyle{remark}
\newtheorem{interpretation}{Interprétation}
\newtheorem*{interpretation*}{Interprétation}

\theoremstyle{remark}
%\newtheorem{remarque}{Remarque}%[section]
%\newtheorem*{remarque*}{Remarque}
\newtheorem{remark}{Remark}%[section]
\newtheorem*{remark*}{Remark}
%\newtheorem{comment}{Comment}%[section]
%\newtheorem*{comment*}{Comment
%\usepackage{mdframed}
%\newtheorem{step}{Step}[section]
%\newtheorem{rem}{Comment}[section]
%\newtheorem{ex}{Example}[section]
%\newtheorem{hist}{History}[section]
%\newtheorem*{ex*}{Example}
\theoremstyle{plain}
\newtheorem{propriete}{Propri\'et\'e}[section]
\newtheorem{lemma}{Lemma}
\newtheorem*{lemma*}{Lemma}
\newtheorem{theorem}{Theorem}
\newtheorem*{theorem*}{Theorem}
\theoremstyle{definition}
\newtheorem{definition}{Définition}%[section]
\theoremstyle{definition}

%\newtheorem{definitionx}{D\'efinition}%[section]
%\newenvironment{definition}
 %{\pushQED{\qed}\renewcommand{\qedsymbol}{$\triangle$}\definitionx}
 %{\popQED\enddefinitionx}

\newtheorem{condition}{Condition}
\renewcommand{\thecondition}{C\arabic{condition}}
\theoremstyle{remark}
%\newtheorem{exemple}{Exemple}
%\newtheorem*{exemple*}{Exemple}
\newtheorem{example}{Example}
\newtheorem*{example*}{Example}
\newtheorem{remarque}{Remarque}
\newtheorem*{remarque*}{Remarque}
\newtheorem{exemple}{Exemple}
\newtheorem*{exemple*}{Exemple}
%\newtheorem{example}{Example}[section]
%\newtheorem{remark}{Remark}[section]
%\newtheorem{notation}{Notation}[section]
%\newtheorem{assumption}{Assumption}[section]
%\newtheorem{condition}{Condition}[section]
%\newtheorem{exercise}{Ex.}[section]
%\newtheorem{fact}{Fact}[section]
\newenvironment{prt}
  {
  \begin{mdframed}
  \begin{propriete}}
  {\end{propriete}
  \end{mdframed}
  }
  
  \newenvironment{dft}
  {
  \begin{mdframed}\begin{definition}}
  {\end{definition}\end{mdframed}
  }



%\newenvironment{example*}
 % {\begin{mdframed}\begin{ex*}}
  %{\end{ex*}\end{mdframed}}


\usepackage{xcolor}
 \colorlet{linkequation}{ugablue} 
 
\newcommand*{\SavedEqref}{}
 \let\SavedEqref\eqref 
\renewcommand*{\eqref}[1]{%
\begingroup \hypersetup{
      linkcolor=linkequation,
linkbordercolor=linkequation, }%
\SavedEqref{#1}%
 \endgroup
}

\newcommand*{\refeq}[1]{%
 \begingroup
\hypersetup{ 
linkcolor=linkequation, 
linkbordercolor=linkequation,
}%
\ref{#1}%
 \endgroup
}


\usepackage{tikz}
\usepackage{enumitem}


\usepackage{fancyhdr}
\pagestyle{fancy}
%\renewcommand{\subsection{mark}[1]{\markright{#1}{}}
\fancyhead{}
\fancyfoot{} 
%\fancyhead[LE,LO]{\tiny{\thepage}}
\fancyhead[C]{\small\textsc{Économétrie, L3 Miash, S2}}
%fancyhead[CE,CO]{\tiny{\rightmark}}
\fancyhead[L]{\small\textsc{UGA, FEG}}
\fancyfoot[C]{\small{\thepage}}
%\fancyfoot[R]{\small \textcopyright \ \  \small\textsc{Michal W. Urdanivia}}
\fancyhead[R]{ \small\textsc{M. Urdanivia}}
%\renewcommand{\headrulewidth}{0pt}
\renewcommand{\footrulewidth}{0pt}

%\pagenumbering{roman}


\begin{document} 
\usetikzlibrary{positioning}
\usetikzlibrary{snakes}
\usetikzlibrary{calc}
\usetikzlibrary{arrows}
\usetikzlibrary{decorations.markings}
\usetikzlibrary{shapes.misc}
\usetikzlibrary{shapes}
%\tikzset{block/.style={draw, rectangle, fill=gray!20, text centered, minimum height = 3em,text width=7em}}
%\tikzset{empty/.style={draw, rectangle, fill=none, text centered, minimum height = 3em,text width=7em}}
%\tikzset{line/.style={draw, -latex'}}
%\onehalfspace

\begin{titlepage}
\centering
	%\includegraphics[width=0.15\textwidth]{logoUGA2020.pdf}\par\vspace{1cm}
	{\scshape\Large \textsc{Économétrie: S2, UGA, L3 Miash}\par}
	\vspace{1cm}
	%{\scshape\large \textsc{Extremum Estimators(1)}\par}
	%\vspace{1cm}
	{\Large\bfseries \textsc{2: Régression Linéaire et Moindres Carrés} \par}
    \vspace{1cm}   
	{\Large\bfseries \textsc{Analyse et Résultats Asymptotiques} \par}
	\vspace{1cm}
	{(\textsc{Cette version: \today})\par}
	\vspace{1cm}
	{\large \textsc{Michal Urdanivia}
	\footnote{Contact:  
	\href{mailto:michal.wong-urdanivia@univ-grenoble-alpes.fr}{michal.wong-urdanivia@univ-grenoble-alpes.fr}, 
	 Université de Grenoble Alpes,  Faculté d'\'Economie, GAEL.}\par}
	%\vfill
	%supervised by\par
	%Dr.~Mark \textsc{Brown}
%\vfill
% Bottom of the page
	
\end{titlepage}

\newpage

\tableofcontents

\newpage

\section{Le modèle}
On s'intéresse à la relation entre une variable $Y\in \mathbb{R}$, appelée \emph{variable dépendante}, et un vecteur $X\in \mathbb{R}^K$, de variables appelées  régresseurs. Pour cela nous disposons de données $\{(Y_i, X_i)\}_{i=1}^n$, et le modèle que nous considérons est un modèle de régression linéaire défini par les hypothèses suivantes.
\begin{condition}
Les données  $\{(Y_i, X_i), i = 1,...,n\}$ sont un échantillon i.i.d.
\label{cond1}
\end{condition}
\begin{condition} $Y_i$  et $X_i$ vérifient,
\begin{align*} 
Y_i= X_i^\top\beta + U_i \ \ i = 1,...,n
\end{align*}
où $U_i$ est une variable inobservée(ou terme d'erreur) vérifiant $\Exp(U_i) = 0$.
\label{cond2} 
\end{condition}
\begin{condition}$X_i$ est (faiblement)exogène\footnote{On pourrait aussi supposer une \emph{exogénéité forte} avec $\Exp(U_i|X_i) = 0$
    auquel cas $\beta$ dans $Y_i = X_i^\top\beta + U_i$ seraient les paramètres d'un modèle de régression linéaire, à savoir d'un modèle tel 
    que $\Exp(Y_i|X_i) = X_i^\top \beta $.} par rapport à $U_i$,
\begin{align*}
\Exp(X_iU_i) = 0
\end{align*}
\label{cond3} 
\end{condition}
\begin{condition} 
La matrice $\Exp(X_iX_i^\top)$ est finie et définie positive.
\label{cond4}
\end{condition}
\begin{condition}
$\Exp(X_{i,k}^4) < \infty$ , pour tout $k=1,...,K$.
\label{cond5}
\end{condition}
\begin{condition}
$\Exp(U_i^4) < \infty$
\label{cond6}
\end{condition}
\begin{condition}
$\Exp(U_i^2X_iX_i^\top)$ est définie positive.
\label{cond7}
\end{condition}

\section{Convergence}
Rappelons qu'une écriture de l' estimateur des moindres carrés est,
\begin{align*}
\hat{\beta}_n= \left(\sumobs X_iX_i^\top\right)^{-1}\sumobs X_iY_i
\end{align*}
Cet estimateur est convergent pour $\beta$ si $\hat{\beta}_n\limp \beta$ quand $n\to \infty$, ce qui est établi par la propriété suivant.
\begin{propriete}(\textbf{Convergence de l'estimateur des moindres carrés})
Sous les hypothèses \ref{cond1} - \ref{cond4}, $\hat{\beta}_n\limp \beta$ quand $n\to \infty$.
\end{propriete}
\begin{proof}
$\hat{\beta}_n$ peut s'écrire,
\begin{align}
\hat{\beta}_n= \beta +  \left(n^{-1}\sumobs X_iX_i^\top\right)^{-1}n^{-1}\sumobs X_iU_i
\label{eq1}
\end{align}
Les termes $U_i$'s et les termes $X_iU_i$'s sont i.i.d. sous l'hypothèse \ref{cond1}. 
Dans ce cas, par la loi faible de grands nombres\footnote{
Soit $X$ une variable aléatoire, et définissons, 
\begin{align*}
    X^{+} &=\max(0, X)\\ 
    X^{-} &=\max(0, -X),
\end{align*}
de sorte que, 
\begin{align*} 
    X &= X^{+} - X^{-}.
\end{align*}
On note que $X^{+}$ et $X^{-}$ sont toutes les deux des variables aléatoires positives. 
Quand au moins une des conditions suivantes est satisfaite: 
\begin{enumerate}[label = (\roman*)]
    \item $\Exp(X^{+}) < \infty$, ou,
    \item $\Exp(X^{-}) < \infty$,
\end{enumerate}
la valeur espérée de $X$ est donnée par, 
\begin{align*} 
    \Exp(X) &= \Exp(X^{+}) - \Exp(X^{-}).
\end{align*}
$\Exp(X)$ n'est pas définie quand  $\Exp(X^{+}) = \infty$ et $\Exp(X^{-})=\infty$(ainsi nous excluons/interdisons 
$\infty - \infty$. Comme, 
\begin{align*} 
    \abs{X} &= X^{+}+ X^{-},
\end{align*}
nous avons que $\Exp(\abs{X})< \infty$ si et seulement si $\Exp(X^{+}) <\infty$ et
$\Exp(X^{-}) <\infty$. Quand nous disons que $\Exp(X) = \mu$ pour un certain $\mu$, nous supposons donc que  soit
$\Exp(X^{+}) <\infty$ ou $\Exp(X^{-}) <\infty$ pour que $\Exp(X)$ soit définie.
Si $\mu$ est fini, il doit être vrai que $\Exp(X^{+}) < \infty$ et $\Exp(X^{-}) <\infty$, et
par conséquent que $\Exp(\abs{X}) < \infty$.
},
\begin{align*}
n^{-1} \sumobs X_iU_i \limp \Exp(X_iU_i) = 0
\end{align*}
Où l'on utilise \ref{cond3}. Dans la mesure où $E(X_iX_i^\top)$ est finie sous l'hypothèse \ref{cond4} nous avons par la loi faible des grand nombres,
\begin{align*}
n^{-1} \sumobs X_iX_i^\top \limp \Exp(X_iX_i^\top)
\end{align*}
et comme  $E(X_iX_i^\top)$ est définie positive, nous avons  par le théorème de Slutsky,
\begin{align}
\left(n^{-1} \sumobs X_iX_i^\top\right)^{-1} \limp \left(\Exp(X_iX_i^\top)\right)^{-1}
\label{eq2}
\end{align}
Et par conséquent,
\begin{align*}
\left(n^{-1}\sumobs X_iX_i^\top\right)^{-1}n^{-1}\sumobs X_iU_i \limp 0
\end{align*}
et donc,
\begin{align*}
\hat{\beta}_n\limp \beta
\end{align*}
\end{proof}

\section{Distribution asymptotique}
Le résultat suivant établit la distribution asymptotique de l'estimateur des moindres carrés.
\begin{propriete}\label{pr2}(\textbf{Normalité asymptotique})
Sous les hypothèses \ref{cond1}-\ref{cond7},
\begin{align*}
n^{1/2}(\hat{\beta}_n- \beta) \limd \mathcal{N}(0, \Vr)
\end{align*}
où
\begin{align*}
\Vr = Q^{-1} \Omega Q^{-1}, \ \ Q = \Exp(X_iX_i^\top), \ \ \Omega =  \Exp(U_i^2X_iX_i^\top)
\end{align*}
\end{propriete}
\begin{proof}
Nous avons en utilisant \eqref{eq1},
\begin{align*}
n^{1/2}(\hat{\beta}_n- \beta) = \left(n^{-1}\sumobs X_iX_i^\top\right)^{-1}n^{-1/2}\sumobs X_iU_i 
\end{align*}
Commençons en considérant le terme $n^{-1/2}\sumobs X_iU_i$. Par \ref{cond3}, $\Exp(X_iU_i) = 0$. Ensuite, considérons 
$\Var(X_iU_i) = \Exp(U_i^2X_iX_i^\top)$.    
L'élément  $(r, s)$, $r,s = 1,...,K$, de $\Var(X_iU_i) = \Exp(U_i^2X_iX_i^\top)$ est $\Exp(U_i^2X_{i,r}X_{i,s})$.  
Par l'inégalité de Cauchy-Schwartz et sous les hypothèses \ref{cond5} , \ref{cond6},
\begin{align*}
\Exp\left(\abs{U_i^2 X_{i,r}  X_{i,s}}\right) \leq \left[\Exp(U_i^4)\Exp(X_{i,r}^2X_{i,s}^2)\right]^{1/2} \leq  \left[\Exp(U_i^4)^{1/2}
\Exp(X_ {i,r}^4)\Exp(X_{i,s}^4))\right]^{1/4} < \infty 
\end{align*}
Par le théorème central-limite,
\begin{align}
n^{-1/2}\sumobs X_iU_i  \limd \mathcal{N}\left(0, \Exp(U_i^2X_iX_i^\top)\right) = \mathcal{N}(0, \Omega)
\label{eq3}
\end{align}
Finalement \eqref{eq2}, \eqref{eq3} et le théorème de convergence de Cramer(son extension multivariée) impliquent que,
\begin{align*}
\left(n^{-1}\sumobs X_iX_i^\top\right)^{-1}n^{-1/2}\sumobs X_iU_i \limd Q^{-1} \mathcal{N}(0, \Omega) = \mathcal{N}(0, Q^{-1}\Omega Q^{-1})
\end{align*}
\end{proof}
\begin{remarque}\label{re1}
Les hypothèses de la propriété \ref{pr2} n'excluent pas le cas où la variance conditionnelle des $U_i$'s est une fonction de $X_i$, i.e. il est possible que les termes d'erreur $U_i$'s soient \emph{hétéroscédastiques}: $\Exp(U_i^2|X_i) = \sigma^2(X_i)$ pour une fonction $\sigma^2: \mathbb{R}^K \mapsto \mathbb{R}$.
\end{remarque}
\begin{remarque}\label{re2}
La matrice de variances-covariances asymptotique de $\hat{\beta}_n$ est donnée par la formule "en sandwich",
\begin{align*}
\Vr = \left(\Exp(X_iX_i^\top)\right)^{-1} \Exp(U_i^2X_iX_i^\top) \left(\Exp(X_iX_i^\top)\right)^{-1}
\end{align*}
Si nous imposons la condition que $\Exp(U_i^2|X_i) = \sigma^2$, alors $V$ se simplifie en la matrice des variances-covariances \emph{homoscédastique},
\begin{align}
\Vr = \sigma^2\left(\Exp(X_iX_i^\top)\right)^{-1}
\label{eq4}
\end{align}
En effet par la règle des espérances itérées,
\begin{align*}
\Exp(U_i^2X_iX_i^\top) = \Exp\left(\Exp(U_i^2X_iX_i^\top) | X_i) \right) = \Exp\left(X_iX_i^\top\Exp(U_i^2|X_i)\right) = \sigma^2\Exp(X_iX_i^\top)
\end{align*}
ainsi dans ce cas, 
\begin{align*}
\Omega = \sigma^2 Q \ \ \textrm{et,} \ \ \Vr = Q^{-1}\Omega  Q^{-1} = \sigma^2  Q^{-1} =\sigma^2\left(\Exp(X_iX_i^\top)\right)^{-1}
\end{align*}
\end{remarque}

\section{Estimation de la matrice des variances-covariances}
A partir d'un estimateur de $\beta$, nous pouvons construire les résidus $\hat{U}_i = Y_i - X_i^\top \hat{\beta}_n$.  Considérons l'estimateur suivant de $V$ obtenu par application du principe d'analogie,
\begin{align*}
\hat{\Vr}_n = \hat{Q}_n^{-1} \hat{\Omega}_n \hat{Q}_n^{-1}
\end{align*}
où,
\begin{align*}
\hat{Q}_n = n^{-1}\sumobs X_iX_i^\top \ \ , \ \ \hat{\Omega}_n = n^{-1}\sumobs \hat{U}_i^2 X_iX_i^\top
\end{align*}
Nous avons déjà montré que $\hat{Q}_n ^{-1} \limd Q^{-1}$(c.f., \eqref{eq2}). Considérons maintenant $\hat{\Omega}_n$. Nous pouvons écrire ici,
\begin{align*}
 \hat{U}_i  = U_i - X_i(\hat{\beta}_n- \beta)
\end{align*}
Il en résulte que,
\begin{align}
n^{-1}\sumobs \hat{U}_i^2 X_iX_i^\top = n^{-1} \sumobs U_i^2X_iX_i^\top - 2R_{1,n} + R_{2,n}
\label{eq5}
\end{align}
où,
\begin{align*}
R_{1,n} = n^{-1} \sumobs \left((\hat{\beta}_n- \beta)X_iU_i\right)X_iX_i^\top \ \  , \ \ 
R_{2,n} = n^{-1} \sumobs \left((\hat{\beta}_n- \beta)X_i\right)^2X_iX_i^\top
\end{align*}
Sous les hypothèses de la propriété \ref{pr2}, $\Exp(U_i^2X_iX_i^\top)$ est finie, comme cela a été montré dans la démonstration de la propriété. Par conséquent, par la loi faible des grand nombres,
\begin{align*}
n^{-1} \sumobs U_i^2 X_iX_i^\top \limp \Exp(U^2X_iX_i^\top)
\end{align*}
En outre, il est possible de montrer que $R_{1,n}$ et $R_{2,n}$ convergent en probabilité vers zéro(c.f., annexe) de sorte que,
\begin{align*}
\hat{\Vr}_n \limp \Vr
\end{align*}
L'estimateur de la matrice des variances-covariances $\hat{\Vr}_n = \hat{Q}_n^{-1} \hat{\Omega}_n  \hat{Q}_n^{-1}$, qui est ainsi donné par une formule "en sandwich" est un estimateur convergent que les termes d'erreur soient homoscédastiques ou hétéroscédastiques. Il est fréquent de l'appeler \emph{estimateur convergent robuste à l'hétéroscédasticité}, ou \emph{estimateur robuste de White}(car il fut suggéré par \citep{white1980})

\section{Intervalles de confiance asymptotiques}
Dans cette section nous nous intéressons aux intervalles de confiance pour les éléments de $\beta$. Considérons l'intervalle de confiance suivant pour $\beta_k$, $k=1,...,K$,
\begin{align*}
\CI_{n, k, 1-\alpha} = \left[\hat{\beta}_{n, k} - z_{1-\alpha/2}\sqrt{\left[\hat{\Vr}_n\right]_{k,k} \Big{/} n}, 
\hat{\beta}_{n, k} + z_{1-\alpha/2}\sqrt{\left[\hat{\Vr}_n\right]_{k,k} \Big{/} n}\right]
\end{align*}
où $ z_{1-\alpha/2}$ est le quantile $1-\alpha/2$ de la distribution normale standard et $\left[\hat{\Vr}_n\right]_{k,k} $ est l'élément $(k, k)$ de la matrice $\hat{\Vr}_n$. Nous allons montrer que $\Prob\left(\beta_k \in \CI_{n, k, 1-\alpha} \right) \rightarrow 1 - \alpha$ lorsque $n\rightarrow \infty$. Comme $n^{1/2}(\hat{\beta}_n- \beta) \limd \mathcal{N}(0, V)$, et $\hat{\Vr}_n \limp V$, 
il résulte des théorèmes de convergence Slutsky et de Cramer que,
\begin{align*}
\hat{\Vr}_n^{-1/2} n^{1/2}(\hat{\beta}_n- \beta) \limd \Vr^{-1/2}\mathcal{N}(0, \Vr) = \mathcal{N}(0, \Id_K)
\end{align*}
et par conséquent,
\begin{align*}
\frac{\sqrt{n}(\hat{\beta}_{n, k} - \beta_k)}{\sqrt{\left[\hat{\Vr}_n\right]_{k,k} }} \limd \mathcal{N}(0, 1)
\end{align*}
ce qui peut aussi s'écrire comme,
\begin{align*}
\Prob\left(\frac{\sqrt{n}(\hat{\beta}_{n,k} - \beta_k)}{\sqrt{\left[\hat{\Vr}_n\right]_{k,k} }} \leq z \right) \rightarrow \Prob(Z\leq z) \ \ \textrm{pour tout} \ \ z \in \mathbb{R},
\end{align*}
où $Z$ est une variable aléatoire et $Z \sim \mathcal{N}(0, 1)$. A présent,
\begin{align*}
\Prob(\beta_k \in \CI_{n, k, 1-\alpha} ) = \Prob\left(\frac{\sqrt{n}(\hat{\beta}_{n,k} - \beta_k)}{\sqrt{\left[\hat{\Vr}_n\right]_{k,k} }} \leq z_{1-\alpha/2} \right) \rightarrow \Prob(\abs{Z} \leq z_{1-\alpha/2}  ) = 1 - \alpha
\end{align*}
Considérons, par exemple, le cas avec des termes d'erreur homoscédastiques.  Nous avons vu que dans ce cas $\sqrt{n}(\hat{\beta}_n- \beta) \limd \mathcal{N}\left(0, \sigma^2\left(\Exp(XX^\top)\right)^{-1}\right)$. Comme $s^2 \limp \sigma^2$, la matrice des variances-covariances peut être estimée par $s^2\left(\sumobs X_iX_i^\top\right)^{-1}$. Et l'intervalle de confiance pour $\beta_k$ est alors,
\begin{align*}
\left[\hat{\beta}_{n, k} \pm  z_{1 -  \alpha/2} \sqrt{ \left[s^2\left(n^{-1}\sumobs X_iX_i^\top\right)^{-1} \right]_{k,k} \Big{/} n}\right] = \left[\hat{\beta}_{n, k}  \pm z_{1-\alpha/2} \sqrt{\left[s^2\left(\mathbf{X}^\top\mathbf{X}\right)\right]_{k,k}}\right] 
\end{align*}
qui est le même intervalle de confiance que celui à distance finie, sauf qu'on utilise ici les quantiles de la distribution normale standard plutôt que ceux de la loi de student.

\section{Tests d'hypothèses}
Dans cette section nous considérons les tests asymptotiques de l'hypothèse $H_0: h(\beta) = 0$ contre l'alternative 
 $H_1: h(\beta) \neq 0$, où $h: \mathbb{R}^K \mapsto \mathbb{R}^q$ est une fonction continument dérivable dans un voisinage de $\beta$. La contrainte sous $H_0$ inclut le cas des contraintes linéaires de la forme $h(\beta) = \mathbf{R}\beta - r$, où $\mathbf{R}$ est une matrice $q\times K$ et $r$ est un vecteur de taille $q$.\\ 
Considérons la \emph{statistique de test de Wald},
\begin{align*}
 W_n = nh(\hat{\beta}_n)^\top \left(\hat{\asyvar}\left(h(\hat{\beta}_n)\right)\right)^{-1}h(\hat{\beta}_n) = n h(\hat{\beta}_n)^\top\left(\frac{\delta h}{\delta \beta ^ \top}(\hat{\beta}_n)\hat{\Vr}_n \frac{\delta h}{\delta \beta}(\hat{\beta}_n)^\top \right)^{-1} h(\hat{\beta}_n)
\end{align*}
où $\asyvar$ désigne la variance asymptotique. Le test asymptotique de taille $\alpha$ de $H_0: h(\beta) = 0$ est alors défini par la règle,
\begin{align*}
\textrm{Rejeter} \ \ H_0 \ \ \textrm{si} \ \ W_n > \chi^2_{q, 1- \alpha}
\end{align*}
où $\chi^2_{q, 1- \alpha}$ est le quantile $(1-\alpha)$ de la distribution du $\chi^2_q$. Un test s'appuyant sur  $W_n$ est dit convergent si $\Prob(W_n > \chi^2_{q, 1- \alpha} | H_1) \rightarrow 1$.
\begin{propriete}\label{pr3}
Sous les hypothèses \ref{cond1}-\ref{cond6},
\begin{enumerate}
\item\label{pr3a} $\Prob(W_n > \chi^2_{q, 1- \alpha} | H_0) \rightarrow \alpha$.
\item\label{pr3b} $\Prob(W_n > \chi^2_{q, 1- \alpha} | H_1) \rightarrow 1$.
\end{enumerate}
\end{propriete}
\begin{proof} 
\begin{enumerate}
\item Comme $n^{1/2}(\hat{\beta}_n- \beta) \limd \mathcal{N}(0, \Vr)$ et que $h(.)$ est continue en $\beta$, sous $H_0$, et en appliquant la méthode delta,
\begin{align*}
n^{1/2}h(\hat{\beta}_n) \limd \mathcal{N}\left(0, \frac{\delta h}{\delta \beta^\top}(\beta) \Vr \frac{\delta h}{\delta \beta}(\beta)^\top \right)
\end{align*}
En outre,
\begin{align*}
\frac{\delta h}{\delta \beta^\top}(\hat{\beta}_n) \limp \frac{\delta h}{\delta \beta^\top}(\beta) \ \ \textrm{et}, \ \ \hat{\Vr}_n \limp \Vr
\end{align*}
Par la propriété de convergence de Cramer, sous $H_0$,
\begin{align*}
\left(\frac{\delta h}{\delta \beta ^ \top}(\hat{\beta}_n)\hat{\Vr}_n \frac{\delta h}{\delta \beta}(\hat{\beta}_n)^\top \right)^{-1/2}n^{1/2} h(\hat{\beta}_n) 
\limd & \left(\frac{\delta h}{\delta \beta^\top}(\beta) \Vr \frac{\delta h}{\delta \beta}(\beta)^\top\right)^{-1/2} \mathcal{N}\left(0, \frac{\delta h}{\delta \beta^\top}(\beta) \Vr \frac{\delta h}{\delta \beta}(\beta)^\top\right)\\
= & \mathcal{N}\left(0, \Id_q \right)
\end{align*}
Et par la propriété des applications continues, sous $H_0$,
\begin{align*}
W_n \limd \chi^2_q
\end{align*}
ce qui complète la démonstration du point \ref{pr3a} de la propriété.
\item Sous l'hypothèse alternative, $h(\beta)\neq 0$, par le théorème de Slustsky,
\begin{align*}
h(\hat{\beta}_n)\limp h(\beta) \neq 0
\end{align*}
Par conséquent, 
\begin{align*}
W_n /n \limp h(\beta)^\top\left(\frac{\delta h}{\delta \beta^\top}(\beta) \Vr \frac{\delta h}{\delta \beta^\top}(\beta)^\top \right)^{-1} h(\beta)
\end{align*}
et par conséquent, sous $H_1$, 
\begin{align*}
W_n \rightarrow \infty
\end{align*}
\end{enumerate}
\end{proof}
Remarquons que dans le cas de contraintes linéaires $h(\beta) = R\beta - r$, nous avons:
\begin{align*}
W_n = n\left(R\hat{\beta}_n- r\right)^\top\left(R\hat{\Vr}_nR^\top\right)\left(R\hat{\beta}_n- r\right)
\end{align*}
En outre dans le cas homoscédastique, on peut remplacer $\hat{\Vr}_n$ par $s^2(\mathbf{X}^\top\mathbf{X}/n)^{-1}$. Alors, la statistique de Wald devient,
\begin{align*}
W_n = \left(R\hat{\beta}_n- r\right)^\top\left(s^2R(\mathbf{X}^\top\mathbf{X})^{-1}R^\top\right)^{-1}\left(R\hat{\beta}_n-r\right)
\end{align*}
qui est similaire à l'expression de la statistique de Fisher, mis à part l'ajustement relatif au nombre de degrés de liberté dans le numérateur.


\newpage


\appendix

\section{Résultats asymptotiques utilisés}

Nous résumons les principaux résultats utilisés dans cette partie du cours en supposant connus les principaux concepts de 
convergence de suites de variables aléatoires. Pour une suite de variables aléatoires 
$\{X_n: n = 1, 2\ldots\}$ et une variable aléatoire ou non $X$ on note: $X_n\limp X$ 
la convergence en probabilité, $X_n\limps X$ la convergence presque sûre, la $X_n\limd X$
convergence en loi, $X_n\limlp X$ la convergence en moyenne d'ordre $p$.

\begin{theoreme}(\textbf{Loi faible des grands nombres(LFGN)})
    Soit $X_1, \ldots, X_n$ une suite de variables aléatoires i.i.d., telles que $\Exp(\abs{X_i}) <\infty$. Alors, 
    $n^{-1}\sumobs X_i\limp \Exp(X_i)$ quand $n\to\infty$.
\end{theoreme}

\begin{theoreme}(\textbf{de Slutsky})
Supposons que $X_n \limp c$, pour une constante $c$, et soit $h(\cdot)$ une fonction continue en $c$. Alors, 
$h(X_n)\limp h(c)$.
\end{theoreme}

\begin{theoreme}(\textbf{de convergence de Cramer})
    \begin{enumerate}[label = (\roman*)]
    \item Supposons que $X_n \limd X$, et $Y_n \limp c$. alors,
    \begin{enumerate}[label = (\alph*)]
        \item $X_n + Y_n \limd X + c$.
        \item $Y_nX_n \limd cX$,
        \item $X_n/Y_n \limd X/c$, pour $c\neq 0$.
    \end{enumerate}
    \textbf{Remarque}: des résultats similaires sont établis pour des vecteurs/matrices avec 
les définitions appropriés pour la multiplication et division de vecteurs/matrices.
    \item  Si $X_n \limp X$, alors $X_n \limd X$. L'inverse n'est pas vrai:
    \item Si $X_n \limd C$, une constante, alors $X_n \limp C$.
    \item  Si $X_n - Y_n \limp 0$, et $Y_n \limd Y$, alors $X_n\limd Y$.
    \end{enumerate}
\end{theoreme}
\newpage

\section{Convergence de l'estimateur de la matrice des variances-covariances(suite)}
Dans cette annexe nous montrons que les termes $R_{1,n}$ et $R_{2,n}$ de l'équation \eqref{eq5} convergent en probabilité vers zéro. La démonstration utilise le résultat suivant appelé \emph{inégalité de Holder}.
\begin{propriete}\textbf{(Inégalité de Hölder)} Soit $X$ et $Y$ deux variables aléatoires. Si $p>1$, $q>1$,  $1/p +1/q =1$,  alors $\Exp(\abs{XY}) \leq \left(\Exp\abs{X}^p\right)^{1/p}
\left(\Exp\abs{Y}^q\right)^{1/q}$.
\end{propriete}
\textbf{Remarque}: Pour $p = q = 2$ nous avons l'inégalité de Cauchy-Schwartz.

La convergence en probabilité vers zéro élément par élément est équivalente à la convergence en probabilité des normes vers zéro. La norme d'une matrice $A$ est donnée par,
\begin{align*}
\norm{A}& = \left(\Tr(A^\top A)\right)^{1/2}\\
&= \left(\underset{i}{\sum}\underset{j}{\sum}a_{ij}^2\right)^{1/2}
\end{align*}
où $a_{ij}$ est l'élément $(i, j)$ de la matrice $A$. Pour $R_{1,n}$,
\begin{align*}
\norm{n^{-1}\sumobs\left((\hat{\beta}_n- \beta)^\top X_iU_i\right)X_iX_i^\top} &\leq 
n^{-1} \sumobs \norm{\left((\hat{\beta}_n- \beta)^\top X_iU_i\right)X_iX_i^\top}\\
 &= n^{-1} \sumobs \Tr\left(U_i^2\left(\left(\hat{\beta}_n- \beta\right)^\top X_i\right)^2 X_iX_i^\top X_iX_i^\top\right)^{1/2}\\
 &=n^{-1} \sumobs \abs{U_i}\abs{\left(\hat{\beta}_n- \beta\right)^\top X_i}\norm{X_i}\Tr(X_iX_i^\top)^{1/2} \\
 &=n^{-1}\sumobs \abs{U_i}\abs{\left(\hat{\beta}_n- \beta\right)^\top}\norm{X_i}^2
\end{align*}

\begin{align*}
\abs{(\hat{\beta}_n- \beta)^\top X_i} \leq \norm{\hat{\beta}_n- \beta}\norm{X_i}
\end{align*}
Par conséquent,
\begin{align*}
\norm{R_{1,n}} \leq \norm{\hat{\beta}_n- \beta}n^{-1} \sumobs \abs{U_i}\norm{X_i}^3
\end{align*}
Par l'inégalité de Holder avec $p=4$ et $q=4/3$,
\begin{align*}
\Exp\left(\abs{U_i}\norm{X_i}^3\right)&\leq (\Exp(\abs{U_i}^4))^{1/4}(\Exp(\norm{X_i}^4))^{3/4}\\
&< \infty
\end{align*}
étant donné que par l'hypothèse \ref{cond6} nous avons $\Exp(\abs{U_i})^4<\infty$, et,
\begin{align}
\Exp(\norm{X_i}^4) &= \Exp\left(\underset{j=1}{\overset{K}{\sum}}X_{i, j}^2\right)^2\nonumber\\
&= \underset{j=1}{\overset{K}{\sum}}\underset{k=1}{\overset{K}{\sum}}\Exp(X_{i, j}^2X_{i, k}^2)
\label{eq6}
\end{align}
où $Er(X_{i, j}^2X_{i, k}^2) < \infty$ en raison de l'hypothèse \ref{cond5}, comme cela a été montré dans la propriété \ref{pr2}. Par conséquent, par la LFGN,
\begin{align*}
n^{-1} \sumobs \abs{U_i}\norm{X_i}^3 \limp \Exp(\abs{U_i}\norm{X_i}^3)
\end{align*}
et comme nous avons $\norm{\hat{\beta}_n- \beta} \limp 0$, nous avons que $R_{1,n} \limp 0$.\\
Considérons maintenant le cas de $R_{2,n}$. Par des arguments similaires aux précédents, nous pouvons borner $R_{2,n}$ par,
\begin{align*}
\norm{n^{-1} \sumobs \left((\hat{\beta}_n-\beta)^\top X_i\right)^2 X_iX_i^\top} &\leq
n^{-1}\sumobs  \left((\hat{\beta}_n-\beta)^\top X_i\right)^2 \norm{X_i}\Tr(X_iX_i^\top)^{1/2}\\
&= \norm{(\hat{\beta}_n-\beta)}^2n^{-1} \sumobs\norm{X_i}^4
\end{align*}
Et par \eqref{eq6} et la LFGN,
\begin{align*}
n^{-1} \sumobs\norm{X_i}^4 \limp \Exp(\norm{X_i}^4)
\end{align*}
et par conséquent, $R_{2,n} \limp 0$.

\bibliographystyle{jpe}
\bibliography{../Biblio.bib}
 \end{document}